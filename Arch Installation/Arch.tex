\documentclass[12pt]{article}

% <PACKAGES USED>
\usepackage{minted}
\usepackage[margin=1in]{geometry}
\usepackage{mathptmx}
\usepackage{framed}
\usepackage{multicol}
\usepackage{graphicx}
\usepackage[T1]{fontenc}
\usepackage{xcolor}
\usepackage{color}
\usepackage{fancyhdr}
\usepackage{blindtext}
% </PACKAGES USED>

% <INFORMAL SECTION>
\title{Arch Linux Installation (BIOS)}
\author{Ramin Shahrivar}
% </INFORMAL SECTION>

% </HEADER & FOOTERS>
\pagestyle{fancy}
\fancyhf{}
\lhead{Arch Linux Installation}
\rhead{\thepage}
% </HEADER & FOOTERS>

\begin{document}
\maketitle
\tableofcontents
\pagebreak
\thispagestyle{empty}


\section{Setting Our Timedate}
\begin{minted}{bash}
timedatectl set-ntp true
\end{minted}

\section{Disk Partitioning}
First we should decide about in which disk drive we're going to install our arch linux.
\begin{minted}{bash}
lsblk #For printing out our current disk
fdisk /dev/sda
\end{minted}
Instead of `sdb` you should write the name of the disk you want your linux to be installed.
Now we're on a new command prompt called fdisk.
Type in `p` for listing out your partitions.
Type in `n` for creating a new partition, partition number and first sector are in default so press enter.
Type in `d` to format your partitions.
But for the last sector, we want to put the size to create our partition.

\begin{framed}
\begin{itemize}
\item{Boot: +200M}
\item{Swap: 1.5 Times your memory}
\item{Root: Depends on how much program are you going to install on your linux - at least +15G}
\item{Home: Rest of your space - just leave the last sector empty and automatically does that for you}
\end{itemize}
\end{framed}

And last but not least for the fdisk environment:
Type in `w` for writing out your partitions on the device.
Now we're going to make file system on out `root`, `boot` and `home` partition.

\begin{minted}{bash}
mkfs.ext4 /dev/sda1
mkfs.ext4 /dev/sda3
mkfs.ext4 /dev/sda4
\end{minted}

And for our Swap:

\begin{minted}{bash}
mkswap /dev/sda2
swapon /dev/sda2
\end{minted}

\section{Mounting}
Now we're going to mount our `root` partition into `/mnt`.
\begin{minted}{bash}
mount /dev/sda3 /mnt
\end{minted}
Check `/mnt` with `ls` command:
\begin{minted}{bash}
ls /mnt
\end{minted}

\subsection{Making Directories}
\begin{minted}{bash}
mkdir /mnt/home
mkdir /mnt/boot
\end{minted}

\subsection{Mounting Directories}
Now we should mount our `/home` and `/boot` into the file that we've just created.
\begin{minted}{bash}
mount /dev/sda1 /mnt/boot
mount /dev/sda4 /mnt/home
\end{minted}

\section{Updating Pacman Mirrorlist}
First we install `reflector` package.
\begin{minted}{bash}
pacman -S reflector
\end{minted}
Then we should update our mirrorlist based on download speed.\newline\newline
\textit{\textbf{This will increase a huge amount of download speed}}
\begin{minted}{bash}
reflector --latest 200 --sort rate --save /etc/pacman.d/mirrorlist
\end{minted}

\section{Installtion}

\begin{minted}{bash}
pacstrap /mnt base base-devel linux linux-firmware
\end{minted}
This will install our base core arch linux and our linux kernel.
You can also install the `amd-ucode` or if you have a intel cpu `intel-ucode` and this will improve your cpu performance.

\section{Fstab}
This will generate UUID for our partitions.
\begin{minted}{bash}
genfstab -U /mnt >> /mnt/etc/fstab
\end{minted}

\section{Chroot}
Change root into the new system.
\begin{minted}{bash}
arch-chroot /mnt
\end{minted}

\section{Timezone}

\begin{minted}{bash}
ln -sf /usr/share/zoneinfo/Europe/Switzerland /etc/localtime
\end{minted}

\subsection{Hardware Clock Sync}
This will generate `/etc/adjtime`
\begin{minted}{bash}
hwclock --systohc
\end{minted}

\section{Locale}
Uncomment your suited language.

\begin{minted}{bash}
vim /etc/locale.gen
\end{minted}

\subsection{Generate Locale}
\begin{minted}{bash}
locale-gen
\end{minted}


\subsection{Create a Locale Config}
\begin{minted}{bash}
vim /etc/locale.conf
\end{minted}
And put this in it:
\begin{verbatim}
LANG = en_US.UTF-8
\end{verbatim}

\section{Host Name}
\begin{minted}{bash}
vim /etc/hostname
\end{minted}

\subsection{Local IP For Host}
\begin{minted}{bash}
vim /etc/hosts
\end{minted}
And put this in it:

\begin{verbatim}
127.0.0.1	localhost
::1			localhost
127.0.1.1	Username.localdomain	Username
\end{verbatim}

\section{Initramfs (Not Necessary)}
\begin{minted}{bash}
mkinitcpio -P
\end{minted}

\section{Set Password}
\begin{minted}{bash}
passwd
\end{minted}

\section{Boot Loader}
\begin{minted}{bash}
pacman -S grub
\end{minted}

\subsection{Grub}
\begin{minted}{bash}
grub-install --target=i386-pc /dev/sda
\end{minted}

\subsubsection{Grub Config}
\begin{minted}{bash}
grub-mkconfig -o /boot/grub/grub.cfg
\end{minted}

\section{Installing Useful Software}
\noindent\fbox{%
    \parbox{\textwidth}{%

pacman -S networkmanager network-manager-applet wireless-tools wpa-supplicant dialog mtools dosfstools linux-headers cups bluez bluez-utils git pulseaudio pulseaudio-bluetooth pulseaudio-jack pulseaudio-equalizer xdg-utils xdg-user-dirs

    }%
}

\subsection{Enabling Software In Systemctl}
\begin{minted}{bash}
systemctl enable NetworkManager
systemctl enable org.cups.cupsd
\end{minted}

\section{Adding Users}

\begin{minted}{bash}
useradd -mG wheel <Username>
\end{minted}

\subsection{Sudo Permissions / Sudoers}
Uncomment your wheel section in this file:
\begin{minted}{bash}
EDITOR=vim visudo
\end{minted}

\subsection{Set Password For Your User}
\begin{minted}{bash}
passwd <Username>
\end{minted}

\section{Exit Chroot}
\begin{minted}{bash}
exit
\end{minted}

\section{Unmount}
This will unmount all disks:
\begin{minted}{bash}
umount -a
\end{minted}

\section{Reboot and Have Fun!}
\begin{minted}{bash}
reboot
\end{minted}





\end{document}